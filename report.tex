\documentclass[draft, twocolumn]{article}
\usepackage{cite}
\usepackage{url}
\usepackage{catchfilebetweentags}
\usepackage{amssymb}
\usepackage{turnstile}
\usepackage{bbm}
\usepackage[greek, english]{babel}
\usepackage{MnSymbol}
\usepackage{ucs}
\DeclareUnicodeCharacter{8759}{\ensuremath{\squaredots}}
\DeclareUnicodeCharacter{951}{\textgreek{\texteta}}
\DeclareUnicodeCharacter{737}{\ensuremath{^l}}
\DeclareUnicodeCharacter{691}{\ensuremath{^r}}
\DeclareUnicodeCharacter{8718}{\ensuremath{\blacksquare}}
\DeclareUnicodeCharacter{957}{\textgreek{\textnu}}
\DeclareUnicodeCharacter{961}{\textgreek{\textrho}}
\DeclareUnicodeCharacter{954}{\textgreek{\textkappa}}
\DeclareUnicodeCharacter{10214}{\ensuremath{\lsem}}
\DeclareUnicodeCharacter{10215}{\ensuremath{\rsem}}
\usepackage[utf8x]{inputenc}
\usepackage[T1]{fontenc}
\usepackage{autofe}
\usepackage{agda}
\usepackage{bbding}
\newcommand\doubleplus{+\kern-1.3ex+\kern0.8ex}
\newcommand\mdoubleplus{\ensuremath{\mathbin{+\mkern-10mu+}}}
\author{D Oisín Kidney}
\title{An Efficient and Flexible Evidence-Providing Polynomial Solver for
  Polynomials in Agda}
\begin{document}
\maketitle
\begin{abstract}
  We provide an efficient implementation of a polynomial solver in the
  programming language Agda, and demonstrate its use in a variety of
  applications.
\end{abstract}
\tableofcontents
\section{Introduction}
Dependently typed languages such as Agda\cite{norell_dependently_2008} and
Coq\cite{the_coq_development_team_2018_1219885} allow programmers to write
machine-checked proofs as programs. They provide a degree of reassurance that
handwritten proofs cannot, and allow for exploration of abstract concepts in a
machine-assisted environment.

We will describe an efficient implementation of an automated prover for
equalities in ring and ring-like structures, and show how it can be extended for
use in settings more exotic than simple equality.
\section{Monoids}
Before describing the ring solver, first we will explain the simpler case of a
monoid solver.

A monoid is a set equipped with a binary operation, \(\bullet\), and a
distinguished element \(\epsilon\), which obeys the laws:
\begin{align}
  x \bullet (y \bullet z) &= (x \bullet y) \bullet z \tag{Associativity} \\
  x \bullet \epsilon      &= x \tag{Left Identity} \\
  \epsilon \bullet x      &= x \tag{Right Identity}
\end{align}
\subsection{Equality Proofs}
Monoids can be represented in Agda in a straightforward way, as a record (see
figure~\ref{mon-def}).
\begin{figure}
  \ExecuteMetaData[Code.tex]{mon-def}
  \caption{The definition of Monoid in the Agda Standard
    Library\cite{danielsson_agda_2018}}
  \label{mon-def}
\end{figure}

These come equipped with their own equivalence relation, according to which
proofs for each of the monoid laws are provided. Using this, we can prove
identities like the one in figure~\ref{mon-ident}.
\begin{figure}[h]
  \ExecuteMetaData[Code.tex]{mon-ident}
  \caption{Example Identity}
  \label{mon-ident}
\end{figure}

While it seems like an obvious identity, the proof is somewhat involved
(figure~\ref{mon-proof}).
\begin{figure}[!h]
  \ExecuteMetaData[Code.tex]{mon-proof}
  \caption{Proof of identity in figure~\ref{mon-ident}}
  \label{mon-proof}
\end{figure}

The syntax mimics that of normal, handwritten proofs: the successive ``states''
of the expression are interspersed with equivalence proofs (in the brackets).
Perhaps surprisingly, the syntax is not built-in: it's simply defined in the
standard library.

Despite the powerful syntax, the proof is mechanical, and it's clear that
similar proofs would become tedious with more variables or more complex algebras
(like rings). Luckily, we can automate the procedure.
\subsection{Canonical Forms}
Automation of equality proofs like the one above can be accomplished by first
rewriting both sides of the equation into a canonical form. This form depends on
the particular algebra used in the pair of expressions. For instance, a suitable
canonical form for monoids is lists.
\ExecuteMetaData[Code.tex]{list-def}

In the ``language of monoids'', the neutral element and binary operator have
their equivalents in lists: \(\epsilon\) is simply the empty list, whereas
\(\bullet\) is list concatenation.
\ExecuteMetaData[Code.tex]{list-monoid}

Translating between the language of lists and the language of monoids is
accomplished with \(\mu\) and \(\eta\).
\ExecuteMetaData[Code.tex]{list-trans}

We have one half of the equality so far: that of the canonical forms. As such,
we have an ``obvious'' proof of the identity in figure~\ref{mon-ident},
expressed in the list language (figure~\ref{list-obvious}).
\begin{figure}[!h]
  \ExecuteMetaData[Code.tex]{list-obvious}
  \caption{The identity in figure~\ref{mon-ident}, expressed in the list
    language}
  \label{list-obvious}
\end{figure}
Showing that the list language is equivalent to the monoid language is the next
task.

\subsection{Homomorphism}
We need to show that the equality proof above is a valid proof of the same
identity in figure~\ref{mon-ident}. To do that, we will need to prove the
following:
\begin{align}
  (\eta x) \mu           &= x \\
  (x \mdoubleplus y) \mu &= x \mu \bullet y \mu \\
  [] \mu                 &= \epsilon
\end{align}
1 simply states that translation to and from doesn't change anything, while 2
and 3 give us a monoid homomorphism. Once done, we can use a separate AST type,
something potentially like:
\ExecuteMetaData[Code.tex]{mon-ast}

This AST can be evaluated in a straightforward way:
\ExecuteMetaData[Code.tex]{eval-ast}

\section{Horner Normal Form}
\subsection{Sparse}
\section{Multivariate}
\subsection{Sparse}
\subsection{K}
\section{Setoid Applications}
\subsection{Traced}
\subsection{Isomorphisms}
\subsection{Counterexamples}
\section{The Correct-by-Construction Approach}
\section{Reflection}
\bibliographystyle{IEEEtranS}
\bibliography{horners-rule.bib}
\end{document}